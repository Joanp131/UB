\documentclass[a4paper, 11pt]{article}
\usepackage[utf8]{inputenc}
\usepackage{amssymb}
\usepackage{amsmath}
\usepackage{ragged2e} 
\usepackage[catalan]{babel}
\usepackage{fancyhdr}

\title{Deures 24/02/2020}
\author{Joan Pau Condal Marco}
\date{\today}

\pagestyle{fancy}
\fancyhf{}
\rhead{Joan Pau Condal Marco}
\lhead{Deures 24/02}
\rfoot{Page \thepage}

\begin{document}
    \maketitle
    \justify

    \section{Enunciat:}
        Siguin \emph{E} un espai vectorial i ${F\subset E}$ un subespai.
        \begin{enumerate}
            \item Prova que ${f:E \rightarrow E/F}$ definida com a ${f(u) := u + F}$ \'es una aplicaci\'o lineal.
            \item Prova que \emph{f} \'es epimorfisme.
            \item Prova que ${ker(f) = F}$.
        \end{enumerate}

    \section{Demostraci\'o:}
        Per la definici\'o de ${E/F}$ sabem que ${u+F = [u]}$ i recordem tamb\'e les definicions de ${+_Q \text{ i } \cdot_Q}$:
        \begin{gather*}
            [u] +_Q [v] := [u+v]\\
            \alpha\cdot_Q [u] = [\alpha u]\\
            \forall u,v \in E, \alpha \in \mathbb{R}
        \end{gather*}

        \subsection{Prova que \emph{f} \'es aplicaci\'o lineal:}
            Siguin ${u,v \in E}$ i ${\alpha,\beta\in\mathbb{R}}$:
            \begin{equation*}
                f(\alpha u + \beta v) = [\alpha u + \beta v] = [\alpha u] + [\beta v] = \alpha [u] + \beta [v] = \alpha f(u) + \beta f(v) 
            \end{equation*}
            Per tant, queda demostrat que \emph{f} \'es una aplicaci\'o lineal.

        \subsection{Prova que \emph{f} \'es un epimorfisme:}
            Sabem que \emph{f} \'es epimorfisme ${\iff Im(f) = E/F}$.
            Per la definici\'o de ${Im(f)}$ sabem que:
            \begin{equation*}
                Im(f) = \left\{f(u): u\in E\right\} = \left\{[u]: u\in E\right\} = E/F
            \end{equation*}
            Com que ${Im(f) = E/F \implies f}$ \'es epimorfisme.

        \subsection{Prova que el nucli de \emph{f} \'es \emph{F}:}
            Per la definici\'o de \emph{E/F} sabem que ${u\in F \iff [u] = [0]}$.

            \subsubsection{Nucli \'es subconjunt de \emph{F}:}
                Sigui \emph{u} un vector qualsevol de ${ker(f)}$. Sabem que:
                \begin{equation*}
                    u \in ker(f) \implies f(u) = [0] \implies [u] = [0] \implies u\in F \implies ker(f) \subseteq F.
                \end{equation*}

            \subsubsection{\emph{F} \'es subconjunt del nucli de \emph{f}:}
                Sigui \emph{u} un vector qualsevol de \emph{F}. Sabem que ${f(u) = [0]}$ ja que ${u\in F}$. Per tant, ${u \in ker(f) \implies F \subseteq ker(f)}$.\\\\
            Com que ${ker(f) \subseteq F \text{ i } F \subseteq ker(f) \implies F = ker(f)}$.

\end{document}