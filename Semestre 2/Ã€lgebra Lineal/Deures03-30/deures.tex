\documentclass[a4paper, 12pt]{article}

\usepackage[utf8]{inputenc}
\usepackage{amsmath}
\usepackage{amssymb}
\usepackage{fancyhdr}

\usepackage[catalan]{babel}
\usepackage[
    left=0.90in,%
    right=0.90in,%
    top=1.0in,%
    bottom=1.0in,%
    paperheight=11in,%
    paperwidth=8.5in%
]{geometry}

% fancy header & foot
\pagestyle{fancy}
\fancyhf{}

\setlength{\headheight}{20pt}

\fancyhead[R]{Joan Pau Condal Marco}
\lhead{Homework 30/03}

\fancyfoot[R]{Pàg. \thepage}

\renewcommand{\headrulewidth}{0.4pt}
\renewcommand{\footrulewidth}{0.4pt}
% fancy header & foot

\newcommand{\B}{\mathcal{B}}
\newcommand{\R}{\mathbb{R}}
\newcommand{\f}{\varphi}
\renewcommand{\u}{\alpha_1u_1+\cdots+\alpha_nu_n}
\renewcommand{\v}{\beta_1u_1+\dots+\beta_nu_n}

\begin{document}

    \begin{center}
        \Large
        \textbf{Joan Pau Condal Marco\\Homework 30/03\\}
        \normalsize
    \end{center}

    \section*{Apartat 1:}
    Sigui $\B = (u_1,\dots,u_n)$, aleshores
    \begin{equation}
        \label{eq:gram}
        G(\f, \B) = 
        \begin{pmatrix}
            \f(u_1,u_1) & \f(u_1,u_2) & \cdots & \f(u_1,u_n)\\
            \f(u_2,u_1) & \f(u_2,u_2) & \cdots & \vdots \\
            \vdots & \vdots & \ddots & \vdots \\
            \f(u_n, u_1) & \f(u_n,u_2) & \cdots & \f(u_n,u_n)
        \end{pmatrix}
    \end{equation}
    Per demostrar la igualtat de l'enunciat, desenvoluparem els dos termes per separat i comprovarem que obtenim el mateix resultat.\\
    Primer de tot, desenvoluparem la $\f(u,v)$, on $(u)_\B = (\alpha_1,\dots,\alpha_n)$ i $(v)_\B = (\beta_1,\dots,\beta_n)$
    \begin{align*}
        \f(u,v) &= \f(\u,\v)=\\
                &= \f(\u,\v) + \cdots + \f(\u,\v)=\\
                &= \sum_{i=1}^n\f(\alpha_iu_i,\v) = \sum_{i=1}^n\left( \f(\alpha_iu_i,\beta_1u_1)+\cdots+\f(\alpha_iu_i,\beta_nu_n) \right)=\\
                &= \sum_{i=1}^n \sum_{j=1}^n \f(\alpha_iu_i,\beta_ju_j) = \sum_{i=1}^n\sum_{j=1}^n \alpha_i\f(u_i,\beta_ju_j) = \sum_{i=1}^n\sum_{j=1}^n \alpha_i\beta_j\f(u_i,u_j)
    \end{align*}
    Ara, si aconseguim demostrar la seg\"uent igualtat
    \begin{equation}
        \label{eq:fi_u_v}
        \begin{pmatrix}
            \alpha_1 & \cdots & \alpha_n
        \end{pmatrix} 
        G(\f,\B) 
        \begin{pmatrix}
            \beta_1\\
            \vdots\\
            \beta_n
        \end{pmatrix}
        = \sum_{i=1}^n\sum_{j=1}^n \alpha_i\beta_j\f(u_i,u_j)
    \end{equation}
    haurem demostrat el que voliem. Per demostrar la igualtat \eqref{eq:fi_u_v}, anem a fer el producte de matrius.
    \begin{align*}
        G(\f,\B)
        \begin{pmatrix}
            \beta_1\\
            \vdots\\
            \beta_n
        \end{pmatrix}
        =
        \begin{pmatrix}
            \beta_1\f(u_1,u_1) + \cdots + \beta_n\f(u_1,u_n)\\
            \vdots\\
            \beta_1\f(u_n,u_1) + \cdots + \beta_n\f(u_n,u_n)
        \end{pmatrix}
        =
        \begin{pmatrix}
            \sum_{j=1}^n\beta_j\f(u_1,u_j)\\
            \vdots\\
            \sum_{j=1}^n\beta_j\f(u_n,u_j)
        \end{pmatrix}
    \end{align*}
    \begin{align*}
        \begin{pmatrix}
            \alpha_1 & \cdots & \alpha_n
        \end{pmatrix}
        \begin{pmatrix}
            \sum_{j=1}^n\beta_j\f(u_1,u_j)\\
            \vdots\\
            \sum_{j=1}^n\beta_j\f(u_n,u_j)
        \end{pmatrix}
        = \alpha_1 \sum_{j=1}^n\beta_j\f(u_1,u_j) + \cdots + \alpha_n \sum_{j=1}^n\beta_j\f(u_n,u_j) =\\
        = \sum_{i=1}^n\sum_{j=1}^n \alpha_i\beta_j\f(u_i,u_j) = \f(u,v)
    \end{align*}

    \section*{Apartat 2:}
\end{document}