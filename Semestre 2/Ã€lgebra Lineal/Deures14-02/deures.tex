\documentclass[a4paper, 11pt]{article}
\usepackage{amssymb}
\usepackage{amsmath}
\usepackage{ragged2e}
\usepackage{geometry}
 \geometry{
 a4paper,
 total={170mm,250mm},
 top=20mm,
 }
\usepackage[catalan]{babel}

\title{Deures 14/02/2020}
\author{Joan Pau Condal Marco}
\date{\today}

\begin{document}
    \maketitle
    \justify

    \section*{Enunciat:}
        Es sap que si ${\langle B_1\rangle=\langle B_2\rangle}$ i ${\#B_1 < \#B_2}$, aleshores ${B_2}$ \'es linealment dependent.\\
        Prova que si ${B_1}$ i ${B_2}$ s\'on bases de \emph{F}, aleshores ${\#B_1 = \#B_2}$.

    \section*{Demostraci\'o:}
        Per demostrar la proposici\'o anterior, utilitzarem el metode de reducci\'o a l'absurd; per tant, suposarem que si ${B_1}$ i ${B_2}$ s\'on base de \emph{F}, aleshores ${\# B_1 \neq \# B_2}$. Per la desigualtat anterior, obtenim dos casos a demostrar:

        \subsection*{1. ${\# B_1 < \# B_2}$:}
            Si ${\# B_1 < \# B_2}$, sabem que aleshores ${B_2}$ \'es linealment dependent, afirmaci\'o que va en contra de la nostra hip\'otesis inicial i per tant arribem a una \textbf{contradicci\'o}.
            
        \subsection*{2. ${\# B_1 > \# B_2}$:}
            Si ${\# B_1 < \# B_2}$, sabem que aleshores ${B_1}$ \'es linealment dependent, afirmaci\'o que va en contra de la nostra hip\'otesis inicial i per tant arribem a una segona \textbf{contradicci\'o}.\\
    
        Per tant, podem afirmar que si ${B_1}$ i ${B_2}$ s\'on bases de \emph{F}, ${\#B_1 = \#B_2}$.
\end{document}