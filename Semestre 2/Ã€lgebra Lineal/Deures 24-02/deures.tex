\documentclass[a4paper, 9pt]{article}
\usepackage{amssymb}
\usepackage{amsmath}
\usepackage{ragged2e}
\usepackage{geometry}
 \geometry{
 a4paper,
 total={170mm,250mm},
 top=20mm,
 }
\usepackage[catalan]{babel}

\title{Deures 10/02/2020}
\author{Joan Pau Condal Marco}
\date{\today}

\begin{document}
    \maketitle
    \justify
    \section*{Enunciat:}

    Considerem en ${\mathbb{R}^n}$ les operacions definides com:
    \begin{align*}
        \begin{aligned}
            (a_1, \dots , a_n) +^* (b_1, \dots , b_n) := (a_1 +_{\mathbb{R}} b_1 - 1, \dots , a_n +_{\mathbb{R}} b_n -1) \\
            \alpha \cdot^* (a_1, \dots , a_n) := (\alpha \cdot_{\mathbb{R}} (a_1 -1)+1, \dots , \alpha \cdot_{\mathbb{R}} (a_n -1)+1)
        \end{aligned}
    \end{align*}
    Prova que (${\mathbb{R}}^n$, ${+^*}$, ${\cdot^*}$) \'es un espai vectorial. Caracteritza al vector nul de ${\mathbb{R}^n}$, i al vector oposat de ${(a,b)}$.  \\

    \section*{Demostraci\'o:}
    Per demostrar que (${\mathbb{R}}^n$, ${+^*}$, ${\cdot^*}$) \'es un espai vectorial haurem de demostrar que es compleixen les vuit condicions de les dues operacions (${+^*, \cdot^*}$).\\
    Per les demostracions seg\"uents considerarem:
    \begin{align*}
        \begin{array}{lcl}
            a = (a_1,\dots,a_n)& & \\
            b = (b_1,\dots,b_n)& & a, b, c \in \mathbb{R}^n\\
            c = (c_1,\dots,c_n)& &\\
            & \alpha,\beta \in \mathbb{R} &
        \end{array}
    \end{align*}
    
    \subsection*{1. ${+^*}$ \'es associativa:}
        Hem de demostrar que ${(a+^*b)+^*c = a+^*(b+^*c), \forall a,b,c \in \mathbb{R}^n}$. Per definici\'o de ${+^*}$ tenim que:
        \begin{align*}
            \begin{aligned}
                (a+^*b)+^*c &= ((a_1+b_1-1)+c1-1,\dots,(a_n+b_n-1)+c_n-1) =\\ 
                        &= (a_1+b_1-1+c_1-1,\dots,a_n+b_n-1+c_n-1) =\\ 
                        &= (a_1+(b_1+c_1-1)-1,\dots,a_n+(b_n+c_n-1)-1) =\\
                        &= a+^*(b+^*c).
            \end{aligned}
        \end{align*}
        D'on queda demostrada la propietat associativa.

    \subsection*{2. ${+^*}$ \'es commutativa}
        Hem de demostrar que ${a +^* b = b +^* a, \forall a,b \in \mathbb{R}^n}$. Utilitzant la definici\'o de ${+^*}$ trobem que:
        \begin{align*}
            \begin{aligned}
                a +^* b &= (a_1 + b_1 -1, \dots , a_n+b_n-1) =\\
                        &=(b_1+a_1-1, \dots , b_n+a_n-1) = b +^* a.
            \end{aligned}
        \end{align*}
        D'on trobem que ${+^*}$ \'es commutativa.

    \subsection*{3. Vector nul} 
        Per demostrar que existeix un vector nul, hem de trobar ${\mathbf{0} \in \mathbb{R}^n}$ tal que ${a + \mathbf{0} = a, \forall a \in \mathbb{R}^n}$. Sigui ${\mathbf{0} = (b_1,\dots,b_n)}$ aplicant la definici\'o de ${+^*}$ tenim:
        \begin{align*}
            \begin{aligned}
                a + \mathbf{0} &\:= (a_1+b_1-1,\dots,a_n+b_n-1) = (a_1,\dots,a_n) = a \\
                            &\implies  a_1+b_1-1 = a_1, \cdots, a_n+b_n-1 = a_n \\
                            &\implies b_1 = 1, \cdots , b_n = 1\\
                            &\implies \mathbf{0} = (1, \dots, 1)
            \end{aligned}
        \end{align*}
        Del procediment podem veure que es compleix la propietat del vector nul i aquest \'es ${\mathbf{0} = (1,\dots,1)}$
        
    \subsection*{4. Suma de l'invers}
        La quarta propietat que hem de demostrar \'es que ${\forall a \in \mathbb{R}^n}$, es compleix ${a +^* (-1)\cdot^*a = \mathbf{0}}$. Per demostrar-ho aplicarem la deficici\'o de ${+^*}$ i de ${\cdot^*}$:
        \begin{align*}
            \begin{aligned}
                a+^*(-1)\cdot^*a &= a +^* (-1\cdot(a_1-1)+1,\dots,-1\cdot(a_n-1)+1) =\\ 
                                 &= a +^* (-a_1+1+1,\dots,a_n+1+1) =\\ 
                                 &= a +^* (2-a_1,\dots,2-a_n) =\\
                                 &= (a_1+(2-a_1)-1,\dots,a_n+(2-a_n)-1) =\\
                                 &= (1,\dots,1) = \mathbf{0}
            \end{aligned}
        \end{align*}
        De la demostraci\'o observem tamb\'e que el vector invers de un ${a \in \mathbb{R}^n}$ qualsevol \'es ${-a = (2-a_1,\dots,2-a_n)}$.

    \subsection*{5. Propietat distributiva de ${\cdot^*}$:}
        La seg\"uent propietat a demostrar \'es la distributiva del producte: ${\alpha\cdot^*(a+^*b) = \alpha\cdot^* a + \alpha\cdot^* b}$.
        \\Per demostrar-ho, comen\c{c}arem desenvolupant les dues equacions per separat.
        \\\\Primer de tot, desenvoluparem l'equaci\'o ${\alpha \cdot^* (a+^*b)}$. Aplicant la definici\'o de ${+^*}$ obtenim:
        \begin{center}
            ${\alpha\cdot^* (a_1+b_1-1,\dots,a_n+b_n-1)}$
        \end{center}
        I aplicant la definici\'o de ${\cdot^*}$ i desenvolupant arribem a
        \begin{align}
            \begin{aligned}
                &(\alpha\cdot(a_1+b_1-1-1)+1,\dots,\alpha\cdot(a_n+b_n-1-1)+1) = \\
                &= (\alpha\cdot(a_1+b_1-2)+1,\dots,\alpha\cdot(a_n+b_n-2)+1)
            \end{aligned}
        \end{align}
        Tot seguit, desenvoluparem l'equaci\'o ${\alpha\cdot^*a+^*\alpha\cdot^*b}$:
        \begin{align}
            \begin{aligned}
                \alpha\cdot^*a +^* \alpha\cdot^*b &= (\alpha(a_1-1)+1,\dots,\alpha(a_n-1)+1) + (\alpha(b_1-1)+1,\dots,\alpha(b_n-1)+1) = \\
                                                  &= ((\alpha(a_1-1)+1)+(\alpha(b_1-1)+1)-1,\dots,(\alpha(a_n-1)+1)+(\alpha(b_n-1)+1)-1) = \\
                                                  &= (\alpha(a_1+b_1-1-1)+1+1-1,\dots,\alpha(a_n+b_n-1-1)+1+1-1) =\\
                                                  &= (\alpha(a_1+b_1-2)+1,\dots,\alpha(a_n+b_n-2)+1)
            \end{aligned}
        \end{align}
        De l'equaci\'o (1) i (2) obtenim que ${\alpha\cdot^* (a+^*b) = \alpha\cdot^*a +^* \alpha\cdot^*b}$, demostrant la cinquena propietat.

    \subsection*{6. ${(\alpha+\beta)\cdot u= \alpha u + \beta u, \forall u \in E, \forall\alpha,\beta\in\mathbb{R}^n}$:}
        La sisena propietat que hem de demostrar \'es: ${(\alpha+\beta)\cdot u = \alpha u+\beta u, \forall\alpha,\beta\in\mathbb{R}, u\in E}$.
        La demostraci\'o la farem similar a la secci\'o anterior, desenvolupant per separat la igualtat.\\
        Primer de tot desenvoluparem ${(\alpha+_{\mathbb{R}}\beta)\cdot^*u, u = (a_1,\dots,a_n)}$:
        \begin{align}
            \begin{aligned}
                (\alpha+_{\mathbb{R}}\beta)\cdot^*u &= ((\alpha+_{\mathbb{R}}\beta)\cdot(a_1-1)+1,\dots,(\alpha+_{\mathbb{R}}\beta)\cdot(a_n-1)+1) =\\
                                                    &= ((\alpha a_1 -\alpha +\beta a_1 -\beta)+1,\dots,(\alpha a_n -\alpha +\beta a_n -\beta)+1) =
            \end{aligned}
        \end{align}
        I tot seguit desenvoluparem la segona part de l'equaci\'o:
        \begin{align}
            \begin{aligned}
                (\alpha\cdot^*u)+^*(\beta\cdot^*u) &= (\alpha(a_1-1)+1,\dots,\alpha(a_n-1)+1) +^* (\beta(a_1-1)+1,\dots,\beta(a_n-1)+1) =\\
                                                   &= ((\alpha(a_1-1)+1)+(\beta(a_1-1)+1)-1,\dots,(\alpha(a_n-1)+1)+(\beta(a_n-1)+1)-1) =\\
                                                   &= (\alpha a_1 -\alpha +\beta a_1 -\beta +1, \dots,\alpha a_n -\alpha +\beta a_n -\beta +1).
            \end{aligned}
        \end{align}
        Finament, comparant les equacions (3) i (4), veiem que la igualtat es compleix.

    \subsection*{7. Títol 7}
        ${(\alpha\cdot_{\mathbb{R}}\beta)\cdot^*u = \alpha\cdot^*(\beta\cdot^* u)}$\\
        Demostraci\'o:
        \begin{align}
            \begin{aligned}
                \alpha(\beta\cdot^*u) &= \alpha\cdot^*(\beta(a_1-1)+1,\dots,\beta(a_n-1)+1) =\\
                                      &= (\alpha(\beta(a_1-1)+1-1)+1,\dots,\alpha(\beta(a_1-1)+1-1)+1) =\\
                                      &= (\alpha\beta(a_1-1)+1,\dots,\alpha\beta(a_n-1)+1) =\\
                                      &= (\alpha\beta)\cdot^*u.
            \end{aligned}
        \end{align}

    \subsection*{8. Element neutre del producte}
        ${1\cdot^*u = u, \forall u \in \mathbb{R}^n}$.\\
        Demostraci\'o:
        \begin{align*}
            \begin{aligned}
                1\cdot^*u &= (1\cdot(a_1-1)+1,\dots,1\cdot(a_n-1)+1) =\\
                          &= (a_1,\dots,a_n) = u
            \end{aligned}
        \end{align*}

\end{document}  