\documentclass[a4paper, 12pt]{article}

\usepackage[utf8]{inputenc}
\usepackage{amsmath}
\usepackage{amssymb}
\usepackage{fancyhdr}

\usepackage[catalan]{babel}
\usepackage[
    left=0.90in,%
    right=0.90in,%
    top=1.0in,%
    bottom=1.0in,%
    paperheight=11in,%
    paperwidth=8.5in%
]{geometry}

% fancy header & foot
\pagestyle{fancy}
\fancyhf{}

\setlength{\headheight}{20pt}

\fancyhead[R]{Joan Pau Condal Marco}
\lhead{Homework 27/03}

\fancyfoot[R]{Pàg. \thepage}

\renewcommand{\headrulewidth}{0.4pt}
\renewcommand{\footrulewidth}{0.4pt}
% fancy header & foot

\begin{document}

    \begin{center}
        \Large
        \textbf{Joan Pau Condal Marco\\Homework 27/03\\}
        \normalsize
    \end{center}

    \section*{Apartat 1:}
    Siguin $S_1,\dots,S_k$ subespais d'\emph{E}, la suma $S_1+\cdots+S_k$ es defineix com
    \begin{equation*}
        S_1 + \cdots + S_k = \{ u_1+\cdots+u_k, u_i\in S_i, i=1,\dots,k \}
    \end{equation*}
    Anem a demostrar que la suma, amb aquesta definici\'o  \'es un subespai.
    \begin{enumerate}
        \item Considerem $\theta_E$ el vector nul de \emph{E}, sabem per hip\`otesi que $\theta_E \in S_i, i=1,\dots,k$, ja que $S_i$ \'es subespai per tot $i$.\\
              Aleshores, com que $\theta_E = \theta_E + \cdots + \theta_E$, sabem que $\theta_E \in S_1+\cdots+S_k$; demostrant que $S_1+\cdots+S_k \neq \emptyset$.
        \item Considerem $u,v \in S_1+\cdots+S_k$. Els podem descomposar de la seg\"uent manera:
              \begin{align*}
                  &u = u_1 + \cdots + u_k, \; u_i \in S_i, \; i = 1,\dots,k\\
                  &v = v_1 + \cdots + v_k, \; v_i \in S_i, \; i = 1,\dots,k
              \end{align*}
              Aleshores, el vector $u+v$ \'es 
              \begin{align*}
                  u+v &= u_1 + \cdots + u_k + v_1 + \cdots + v_k =\\
                      &= u_1 + v_1 + \cdots + u_k + v_k
              \end{align*}
              Com que $S_i$ \'es subespai, aleshores $u_i + v_i \in S_i,\;i=1,\dots,k$, d'on queda demostrat que $u+v \in S_1+\cdots+S_k$.
        \item Considerem $\alpha \in \mathbb{R}$ i $v \in S_1+\cdots+S_k$. Aleshores, podem escriure el vector $\alpha v$
              \begin{align*}
                  \alpha v &= \alpha\cdot(v_1+\cdots+v_k) = \\
                           &= \alpha v_1 + \cdots + \alpha v_k
              \end{align*}
              I com que $S_1,\dots,S_k$ s\'on subespais, sabem que $\alpha v_i \in S_i,\;i=1,\dots,k$; per tant, $\alpha v \in S_1+\cdots+S_k$.
    \end{enumerate}
    Com que es compleixen les tres propietats, queda demostrat que $S_1+\cdots+S_k$ \'es un subespai vectorial.

    \section*{Apartat 2:}
    Sabent que $S_1+\cdots+S_k$ \'es subespai vectorial, ara hem de demostrar que $S_1+\cdots+S_k = \langle S_1 \cup \cdots \cup S_k \rangle$; on $\langle S_1 \cup \cdots \cup S_k \rangle$ \'es el subespai generat per tots els vectors de la uni\'o $S_1\cup\cdots\cup S_k$.\\
    \\Per demostrar la igualtat, demostrarem les dues inclusions.\\
    \begin{enumerate}
        \item $S_1+\cdots S_k \subseteq \langle S_1 \cup \cdots \cup S_k \rangle$.\\
              Sigui $u \in S_1+\cdots S_k$. Aleshores, $\exists u_i \in S_i, \; i = 1\dots,k : u = u_1+\cdots+u_k$.
              \begin{align*}
                  u_i \in S_i &\implies u_i \in S_1 \cup\cdots\cup S_k,\; \forall i=1,\dots,k.\\
                              &\implies u_i \in \langle S_1 \cup \cdots \cup S_k \rangle \implies u \in \langle S_1 \cup \cdots \cup S_k \rangle
              \end{align*}
        \item $\langle S_1 \cup \cdots \cup S_k \rangle \subseteq S_1+\cdots S_k $.\\
              Sigui $u \in \langle S_1 \cup \cdots \cup S_k \rangle$. Aleshores $\exists \alpha_i \in \mathbb{R}, u_i \in S_i,\; i = 1,\dots,k$ tal que $u = \alpha_1u_1 + \cdots + \alpha_ku_k$. Com que $S_i$ \'es subespai, aleshores sabem que $\exists v_i \in S_i : v_i = \alpha_iu_i, i = 1,\dots,k$. Per tant, podem reescriure $u = v_1 + \cdots + v_k \implies u \in S_1 + \cdots + S_k$.
    \end{enumerate}
    Al demostrar les dues inclusions anteriors, hem demostrat finalment la igualtat inicial.

    \section*{Apartat 3:}
    L'\'ultima demostraci\'o del homework d'avui \'es provar que
    \begin{equation*}
        \bigcap_{\substack{S \supset S_1+\cdots+S_k\\S subespai}} S = S_1 + \cdots + S_k
    \end{equation*}
    Anem a estudiar un moment el conjunt
    \begin{equation*}
        I = \{ S : S \text{ subespai} \text{ i } S_1 + \cdots + S_k \subset S \}
    \end{equation*}
    Podem veure que el mateix subespai $S_1+\cdots+S_k \in I$, ja que com hem demostrat abans \'es subespai, i clarament $S_1 + \cdots + S_k \subset S_1 + \cdots + S_k$.\\\\
    Si estudiem ara el conjunt $I' = I\setminus \{ S_1+\cdots+S_k \}$, \'es f\`acil veure que tots el seus elements seran estrictament majors que $S_1 + \cdots + S_k$; i per definici\'o de $I$, no hi haur\`a cap element de $I'$ que tingui menys elements que $S_1 + \cdots + S_k$.\\\\
    Aleshores, podem veure que tots els elements de $I'$ contindran $S_1 + \cdots + S_k$ i seran estrictament majors; cosa que significa que quan fem la intersecci\'o $\bigcap I$ ens quedar\`a $S_1 + \cdots + S_k$, demostrant aix\'i l'\'ultim apartat.
\end{document}