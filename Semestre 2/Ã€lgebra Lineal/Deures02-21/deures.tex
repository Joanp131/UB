\documentclass[a4paper, 9pt]{article}
\usepackage{amssymb}
\usepackage{amsmath}
\usepackage{ragged2e}
\usepackage{geometry}
 \geometry{
 a4paper,
 total={170mm,250mm},
 top=20mm,
 }
\usepackage[catalan]{babel}

\title{Deures 10/02/2020}
\author{Joan Pau Condal Marco}
\date{\today}

\begin{document}
    \maketitle
    \justify

    \section*{Enunciat:}
        Siguin ${(E_1, +_1, \cdot_1) \text{ i } (E_2, +_2, \cdot_1)}$ espais vectorials, definim en ${E_1 \times E_2}$ les operacions:
        \begin{align*}
            \begin{aligned}
                (u_1, u_2) + (v_1, V_2) := (u_1 +_1 v_1, u_2 +_2 v_2)\\
                @l\cdot (u_1, u_2) := (@l \cdot_1, u_1, @l \cdot_2 u_2)
            \end{aligned}
        \end{align*}
        Demostreu que ${(E_1 \times E_2, +, \cdot)}$ \'es un espai vectorial.

\end{document}