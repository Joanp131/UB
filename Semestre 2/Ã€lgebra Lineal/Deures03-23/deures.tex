\documentclass[a4paper, 11pt]{article}

\usepackage[utf8]{inputenc}
\usepackage{amsmath}
\usepackage{amssymb}
\usepackage{fancyhdr}

\usepackage[catalan]{babel}
\usepackage[
    left=0.90in,%
    right=0.90in,%
    top=1.0in,%
    bottom=1.0in,%
    paperheight=11in,%
    paperwidth=8.5in%
]{geometry}

\pagestyle{fancy}
\fancyhf{}
\rhead{Joan Pau Condal Marco}
\lhead{Homework 23/03}

\rfoot{Pàg. \thepage}

\newcommand{\B}{\mathcal{B}}
\newcommand{\R}{\mathbb{R}}

\begin{document}

    \begin{center}
        \Large
        \textbf{Joan Pau Condal Marco\\Homework 23/03\\}
        \normalsize
    \end{center}

    \section*{Apartats 1 i 2:}
    Siguin
    \begin{align*}
        &A = M_\mathcal{B}(f) = 
        \begin{bmatrix}
            a_1^1 & \cdots & a^1_n\\
            \vdots & \ddots & \vdots\\
            a^n_1 & \cdots & a^n_n    
        \end{bmatrix} \in M(n,n,\R)
        &\text{ i }&
        &B = M_\mathcal{B}(f) - x\cdot I_n = (b^i_j)_{\substack{i = 1,\dots,n\\j = 1,\cdots,n}}
    \end{align*}
    hem de demostrar que
    \begin{align*}
        &p_f(x) := det(f-xI_n) \in \R[x] &\text{ i }& &gr(p_f(x)) = n
    \end{align*}
    Per la definici\'o del polinomi caracter\'istic de \emph{f}, sabem que
    \begin{equation}
        \label{def}
        p_f(x) := det(M_\mathcal{B} - x\cdot I_n) = det(B)
    \end{equation}
    Podem aplicar la definici\'o de determinant a \eqref{def} i obtenim
    \begin{equation}
        \label{def:det}
        p_f(x) = \sum_{\sigma\in S_n}\epsilon_\sigma \prod_{i=1}^n b_i^{\sigma(i)}
    \end{equation}
    On $S_n$ \'es e conjunt de totes les permutacions $\sigma : \{1,\dots,n\} \longrightarrow \{1,\dots,n\}$ i $\epsilon_\sigma$ \'es el signe de la permutaci\'o $\sigma$.\\
    Ara podem descomposar \eqref{def:det} en dos termes, separant $\sigma = id$ de tot el conjunt $S_n$ per obtenir
    \begin{equation}
        \label{def:desc}
        p_f(x) = \prod_{i=1}^n (a^i_i -x) + \sum_{\substack{\sigma \in S_n\\ \sigma \neq id}} \epsilon_\sigma \prod_{i=1}^n b_i^{\sigma(i)}
    \end{equation}
    Aquesta descomposici\'o ens \'es molt \'util ja que podem veure el polinomi $p_f(x)$ descomposat en una suma de $n!$ productoris (ja que $\#\{S_n\} = n!$). Cada productori generar\`a un polinomi de grau entre zero i n, ambd\'os inclosos. El grau del polinomi generat per cada productori dependr\`a de quants elements de la diagonal de $B$ agafa la $\sigma$ associada a aquell mateix productori.\\
    La descomposici\'o \eqref{def:desc} est\`a feta per veure $p_f(x)$ com a suma de dos termes. El primer terme, el productori associat a $\sigma = id$, genera un polinomi de grau $n$, i a coeficients reals, ja que la matriu pertany a $M(n,n,\R)$. D'aquest primer terme veiem que $gr(p_f(x)) \leq n$. Per demostrar que el grau de $p_f(x)$ \'es $n$, hem d'assegurar que el coeficient de $x^n$ no s'anul·li.\\
    El segon terme de \eqref{def:desc} \'es un sumatori de $n!-1$ polinomis a coeficients reals. \'Es f\`acil veure que cap dels polinomis generats per aquest sumatori ser\`a de grau superior a $n-1$, ja que l'\'unica permutaci\'o que agafa els $n$ elements de la diagonal \'es $\sigma = id$ i no existeix cap permutaci\'o a $S_n$ que agafi exactament $n-1$ elements de la diagonal.\\
    Per tant, com que de \eqref{def:desc} veiem que $p_f(x)$ \'es la suma d'un polinomi de grau $n$ i $n!-1$ polinomis de grau estictament menor que $n-1$, queda demostrat que \'es un polinomi a coeficients reals i de grau $n$.
    \section*{Apartat 3:}
    \subsection*{Apartats 3.1 i 3.2:}
    Per demostrar els apartats 3.1 i 3.2 utilitzarem inducci\'o. Recordem que els coeficients de $x^n$ i $x^{n-1}$ nom\'es depenen del productori dels elements de la diagonal de la matriu $M_\B -x\cdot I_n$; per tant, si considerem els dos polinomis
    \begin{align*}
        &p_f(x) & &(a_1^1 -x)\cdots(a_n^n -x)
    \end{align*}
    Sabem que els seus coeficients de $x^n$ i $x^{n-1}$ seran els mateixos. Sabent aix\`o, farem la demostraci\'o per inducci\'o considerant $p_f(x) = (a_1^1 -x)\cdots(a_n^n -x)$.\\
    \textbf{\underline{Inducci\'o sobre n:}}\\
    \underline{n = 1:}\\
    Per $n = 1$ sabem que $M_\B = (a_1^1)$, per tant, $p_f(x) = det((a^1_1 -x\cdot I_1)) = (a_1^1 -x)$.\\
    Podem observar que es compleixen les dues propietats:
    \begin{align*}
        &p_1 = -1 = (-1)^n\\
        &p_0 = a^1_1 = (-1)^{n-1}\cdot a^1_1 = (-1)^{n-1}\sum_{i = 1}^1 a^i_i = (-1)^{n-1} tr(A)
    \end{align*}
    Suposem ara que les propietats es compleixen per $n-1$\\
    \underline{Per n:}\\
    Considerem $p'_f$ el polinomi de $n-1$ de la forma
    $$
        p'_f(x) = (a_1^1 -x)(a_2^2 -x)\cdots (a_{n-1}^{n-1}) = c_0 + c_1x + \cdots + c_{n-1}x^{n-1}
    $$
    I com que sabem que es compleixen les dues propietats, el podem reescriure de la forma seg\"uent:
    $$
        p'_f(x) = c_0 + \cdots + (-1)^{n-2} tr(A_{n-1})x^{n-2} + (-1)^{n-1}x^{n-1}
    $$
    on $tr(A_{n-1}) = \sum_{i=1}^{n-1}a_i^i$.\\
    I ara podem construir el polinomi de grau \emph{n} a partir de $p'_f$ de la manera seg\"uent:
    $$
        p_f(x) = (a_1^1 -x) \cdots (a_{n-1}^{n-1}-x)(a_n^n -x) = p'_f(x) (a_n^n -x)
    $$
    Substitu\"int $p'_f(x)$ i operant una mica obtenim
    \begin{align*}
        p_f(x) &= (c_0 + \cdots + (-1)^{n-2} tr(A_{n-1})x^{n-2} + (-1)^{n-1}x^{n-1})\cdot(a_n^n -x)\\
        &= d_0 + \cdots + (-1)^{n-1}a_n^n x^{n-1} + (-1)^{n-2} tr(A_{n-1}) x^{n-2} (-x) + (-1)^{n-1} x^{n-1} (-x)\\
        &= d_0 + \cdots + \left[ (-1)^{n-1} a_n^n + (-1)^{n-1} tr(A_{n-1}) \right] x^{n-1} + (-1)^n x^n\\
        &= d_0 + \cdots + \left[ (-1)^{n-1}(a_n^n + tr(A_{n-1})) \right] x^{n-1} + (-1)^n x^n\\
        &= d_0 + \cdots + (-1)^{n-1} tr(A) x^{n-1} + (-1)^n x^n
    \end{align*}
    D'on observem que tamb\'e es compleixen les dues propietats, demostrant que les proposicions 3.1 i 3.2 s\'on certes.
    \subsection*{Apartat 3.3:}
    Per demostrar la propietat del determinant, analitzarem el terme independent dels polinomis del tipus $\prod_{i = 1}^n b_i^{\sigma (i)}$, $\sigma \in S_n$, ja que la suma de tots aquests polinomis amb el seu signe corresponent \'es $p_f(x)$. Depenent de $\sigma$, el polinomi del productori pot tenir un grau o un altre, per\`o \'es f\`acil veure que sigui quin sigui el grau del polinomi, el seu terme independent ser\`a $\epsilon_\sigma\prod_{i = 1}^n a_i^{\sigma(i)}$.\\
    Sabent que el terme independent ser\`a la suma de tots aquests termes independents per tot $\sigma \in S_n$, podem reescriure el terme independent de $p_f(x)$ com
    $$
        p_0 = \sum_{\sigma \in S_n}\epsilon_\sigma\prod_{i=1}^n a_i^{\sigma(i)} = det(A)
    $$
    D'on queda demostrada l'\'ultima propietat.
    \section*{Apartat 4:}
    Sigui $B = C^{-1}MC - x I_n$, podem reorganitzar els seus termes de manera que quedi
    $$
        B = C^{-1}MC - x\cdot I_n = C^{-1}MC - C^{-1}x\cdot I_n C = C^{-1}(M - x\cdot I_n)C
    $$
    I aplicant les propietats dels determinants obtenir
    \begin{align*}
        p_{C^{-1}MC}(x) &= det(B) = det(C^{-1}(M - x\cdot I_n)C) = det(C^{-1})\cdot det(M - x\cdot I_n)\cdot det(C) \\
        &= det(C^{-1}C)\cdot det(M - x\cdot I_n) = det(M - x\cdot I_n) = p_M(x)
    \end{align*}
    D'on queda demostrat que $p_{C^{-1}MC}(x) = p_M(x)$
\end{document}