\documentclass[a4paper, 11pt]{article}

\usepackage[utf8]{inputenc}
\usepackage{amsmath}
\usepackage{amssymb}
\usepackage{fancyhdr}

\usepackage[catalan]{babel}
\usepackage[
    left=0.90in,%
    right=0.90in,%
    top=1.0in,%
    bottom=1.0in,%
    paperheight=11in,%
    paperwidth=8.5in%
]{geometry}

\pagestyle{fancy}
\fancyhf{}
\rhead{Joan Pau Condal Marco}
\lhead{Homework 23/03}

\rfoot{Pàg. \thepage}

\newcommand{\B}{\mathcal{B}}

\begin{document}
    \section*{Apartats 1 i 2:}
    Siguin
    $$
        M_\mathcal{B}(f) = 
        \begin{bmatrix}
            a_1^1 & \cdots & a^1_n\\
            \vdots & \ddots & \vdots\\
            a^n_1 & \cdots & a^n_n    
        \end{bmatrix}
        \text{ i }
        B = M_\mathcal{B}(f) - x\cdot I_n = (b^i_j)_{\substack{i = 1,\dots,n\\j = 1,\cdots,n}}
    $$
    Per la definici\'o del polinomi caracter\'istic de \emph{f}, sabem que
    $$
        p_f(x) := det(M_\mathcal{B} - x\cdot I_n) = det(B)
    $$
    Podem aplicar la definici\'o de determinant a la definici\'o anterior i obtenim
    $$
        p_f(x) = \sum_{\sigma\in S_n}\epsilon_\sigma \prod_{i=1}^n b_i^{\sigma(i)}
    $$
    On $S_n$ \'es e conjunt de totes les permutacions $\sigma : \{1,\dots,n\} \longrightarrow \{1,\dots,n\}$ i $\epsilon_\sigma$ \'es el signe de la permutaci\'o $\sigma$. Ara podem descomposar l'equaci\'o anterior en dos termes de la manera seg\"uent:
    $$
        p_f(x) = \prod_{i=1}^n (a^i_i -x) + \sum_{\substack{\sigma \in S_n\\ \sigma \neq id}} \epsilon_\sigma \prod_{i=1}^n b_i^{\sigma(i)}
    $$
    D'aquesta descomposici\'o podem observar que $p_f(x)$ est\`a format per dos termes. El primer terme \'es $\prod_{i=1}^n (a^i_i -x)$, un polinomi de grau \emph{n}, amb \emph{n} arrels: $a_1^1,\dots,a_n^n$; i el segon terme \'es un sumatori de polinomis de grau estrictament menor que \emph{n-1}, ja que $\sigma = id$ \'es l'\'unica permutaci\'o en $S_n$ tal que el productori de $b_i^{\sigma (i)}$, $i = 1,\dots,n$ genera un polinomi de grau \emph{n} i no existeix cap permutaci\'o que agafi n-1 elements de la diagonal.\\
    Com que el primer terme t\'e coeficient de grau \emph{n} i el segon no en t\'e, podem afirmar que aquest terme no s'anular\`a mai i $p_f(x)$ ser\`a un sumatori d'un polinomi de grau \emph{n} i $n! -1$ polinomis de grau estrictament menor que \emph{n-1}. D'aqu\'i dedu\"im que $p_f(x)$ \'es un polinomi i \'es de grau \emph{n}.
    \section*{Apartat 3:}
    \subsection*{Apartats 3.1 i 3.2:}
    Per demostrar els apartats 3.1 i 3.2 utilitzarem inducci\'o. Recordem que els coeficients de $x^n$ i $x^{n-1}$ nom\'es depenen del productori dels elements de la diagonal de la matriu $M_\B -x\cdot I_n$; per tant, si considerem els dos polinomis
    \begin{align*}
        &p_f(x) & &(a_1^1 -x)\cdots(a_n^n -x)
    \end{align*}
    Sabem que els seus coeficients de $x^n$ i $x^{n-1}$ seran els mateixos. Sabent aix\`o, farem la demostraci\'o per inducci\'o considerant $p_f(x) = (a_1^1 -x)\cdots(a_n^n -x)$.\\
    \textbf{\underline{Inducci\'o sobre n:}}\\
    \underline{n = 1:}\\
    Per $n = 1$ sabem que $M_\B = (a_1^1)$, per tant, $p_f(x) = det((a^1_1 -x\cdot I_1)) = (a_1^1 -x)$.\\
    Podem observar que es compleixen les dues propietats:
    \begin{align*}
        &p_1 = -1 = (-1)^n\\
        &p_0 = a^1_1 = (-1)^{n-1}\cdot a^1_1 = (-1)^{n-1}\sum_{i = 1}^1 a^i_i = (-1)^{n-1} tr(A)
    \end{align*}
    Suposem ara que les propietats es compleixen per $n-1$\\
    \underline{Per n:}\\
    Considerem $p'_f$ el polinomi de $n-1$ de la forma
    $$
        p'_f(x) = (a_1^1 -x)(a_2^2 -x)\cdots (a_{n-1}^{n-1}) = c_0 + c_1x + \cdots + c_{n-1}x^{n-1}
    $$
    I com que sabem que es compleixen les dues propietats, el podem reescriure de la forma seg\"uent:
    $$
        p'_f(x) = c_0 + \cdots + (-1)^{n-2} tr(A_{n-1})x^{n-2} + (-1)^{n-1}x^{n-1}
    $$
    I ara podem construir el polinomi de grau \emph{n} a partir de $p'_f$ de la manera seg\"uent:
    $$
        p_f(x) = (a_1^1 -x) \cdots (a_{n-1}^{n-1}-x)(a_n^n -x) = p'_f(x) (a_n^n -x)
    $$
    Substitu\"int $p'_f(x)$ i operant una mica obtenim
    \begin{align*}
        p_f(x) &= (c_0 + \cdots + (-1)^{n-2} tr(A_{n-1})x^{n-2} + (-1)^{n-1}x^{n-1})(a_n^n -x)\\
        &= d_0 + \cdots + (-1)^{n-1}a_n^n x^{n-1} + (-1)^{n-2} tr(A_{n-1}) x^{n-2} (-x) + (-1)^{n-1} x^{n-1} (-x)\\
        &= d_0 + \cdots + \left[ (-1)^{n-1} a_n^n + (-1)^{n-1} tr(A_{n-1}) \right] x^{n-1} + (-1)^n x^n\\
        &= d_0 + \cdots + \left[ (-1)^{n-1}(a_n^n + tr(A_{n-1})) \right] x^{n-1} + (-1)^n x^n\\
        &= d_0 + \cdots + (-1)^{n-1} tr(A) x^{n-1} + (-1)^n x^n
    \end{align*}
    D'on observem que tamb\'e es compleixen les dues propietats, demostrant que les proposicions 3.1 i 3.2 s\'on certes.
    \subsection*{Apartat 3.3:}
    Sigui \emph{k}, $0 \leq k \leq n, k \neq n-1$, el nombre d'elements de la diagonal de $B = M_\B(f) -x\cdot I_n$ que hi ha al productori $\prod_{i = 1}^n b_i^{\sigma (i)}$, per qualsevol $\sigma \in S_n$. Aleshores, podem veure que el terme independent del polinomi $\prod_{i = 1}^n b_i^{\sigma(i)}$ \'es $\prod_{i = 1}^n a_i^{\sigma(i)}$.\\
    Aplicant aquesta propietat a la definici\'o del polinomi caracter\'istic, podem veure que el seu terme independent v\'e donat per
    $$
        p_0 = \sum_{\sigma \in S_n}\epsilon_\sigma\prod_{i=1}^n a_i^{\sigma(i)} = det(A)
    $$
    \section*{Apartat 4:}
    Sigui $B = C^{-1}MC - x\cdot I_n$, podem reorganitzar els seus termes de manera que quedi
    $$
        B = C^{-1}MC - x\cdot I_n = C^{-1}MC - C^{-1}x\cdot I_n C = C^{-1}(M - x\cdot I_n)C
    $$
    I aplicant les propietats dels determinants obtenir
    \begin{align*}
        p_{C^{-1}MC}(x) &= det(B) = det(C^{-1}(M - x\cdot I_n)C) = det(C^{-1})\cdot det(M - x\cdot I_n)\cdot det(C) \\
        &= det(C^{-1}C)\cdot det(M - x\cdot I_n) = det(M - x\cdot I_n) = p_M(x)
    \end{align*}
    D'on queda demostrat que $p_{C^{-1}MC}(x) = p_M(x)$
\end{document}