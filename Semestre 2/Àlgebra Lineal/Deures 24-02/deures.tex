\documentclass[a4paper, 9pt]{article}
\usepackage{amssymb}
\usepackage{amsmath}
\usepackage{ragged2e}
\usepackage{geometry}
 \geometry{
 a4paper,
 total={170mm,250mm},
 top=20mm,
 }

\title{Deures 10/02/2020}
\author{Joan Pau Condal Marco}
\date{\today}

\begin{document}
    \maketitle
    \justify
    \section*{Enunciat:}

    Considerem en ${\mathbb{R}^n}$ les operacions definides com:
    \begin{center}
        \begin{math}
            (a_1, \dots , a_n) +^* (b_1, \dots , b_n) := (a_1 +_{\mathbb{R}} b_1 - 1, \dots , a_n +_{\mathbb{R}} b_n -1) 
        \end{math}
        \begin{math}
            \alpha \cdot^* (a_1, \dots , a_n) := (\alpha \cdot_{\mathbb{R}} (a_1 -1)+1, \dots , \alpha \cdot_{\mathbb{R}} (a_n -1)+1)
        \end{math}
    \end{center}
    Prova que (${\mathbb{R}}^n$, ${+^*}$, ${\cdot^*}$) \'es un espai vectorial. Caracteritza al vector nul de ${\mathbb{R}^n}$, i al vector oposat d'un dau.  \\

    \section*{Demostraci\'o:}
    Per demostrar que (${\mathbb{R}}^n$, ${+^*}$, ${\cdot^*}$) \'es un espai vectorial haurem de demostrar que es compleixen els vuit condicions de les dues operacions:
    
    \subsection*{1. ${+^*}$ \'es associativa:}
        Hem de demostrar que ${(u+v)+w = u+(v+w), \forall u,v,w \in E}$. Per demostrar-ho agafarem vectors ${u = (a_1, \dots , a_n)}$, ${v = (b_1, \dots , b_n)}$ i ${w = (c_1, \dots , c_n)}$, tots en ${\mathbb{R}^n}$. Per definici\'o de ${+^*}$ tenim que:
        \begin{center}
            \begin{math}
                (u+v)+w = ((a_1+b_1-1)+c1-1,\dots,(a_n+b_n-1)+c_n-1) = (a_1+b_1-1+c_1-1,\dots,a_n+b_n-1+c_n-1) = (a_1+(b_1+c_1-1)-1,\dots,a_n+(b_n+c_n-1)-1) = u+(v+w)
            \end{math}
        \end{center}
        D'on queda demostrada la propietat associativa.

    \subsection*{2. ${+^*}$ \'es commutativa}
        Hem de demostrar que ${u + v = v + u, \forall u,v \in E}$. Per fer-ho, utilitzarem dos vectors qualsevols que anomenarem \emph{u} i \emph{v} (${u,v\in \mathbb{R}^n}$), on
        \begin{center}
            \begin{math}
                u = (a_1, \dots , a_n), 
            \end{math}
            \begin{math}
                v = (b_1, \dots , b_n)
            \end{math}
        \end{center}
        Utilitzant la definici\'o de ${+^*}$ trobem que:
        \begin{center}
            \begin{math}
                u + v = (a_1 + b_1 -1, \dots , a_n+b_n-1) = (b_1+a_1-1, \dots , b_n+a_n-1) = v + u
            \end{math}
        \end{center}
        D'on trobem que ${+^*}$ \'es associativa.

    \subsection*{3. Vector nul} 
        Per demostrar que existeix un vector nul, hem de trobar ${\vec{0} \in \mathbb{R}^n}$ tal que ${u + \vec{0} = u, \forall u \in \mathbb{R}^n}$. Sigui ${u = (a_1,\dots,a_n)}$ un vector qualsevol de ${\mathbb{R}^n}$ i ${\vec{0} = (b_1,\dots,b_n)}$ aplicant la definici\'o de ${+^*}$ tenim:
        \begin{center}
            \begin{math}
                u + \vec{0} = (a_1+b_1-1,\dots,a_n+b_n-1) = (a_1,\dots,a_n) = u \implies  a_1+b_1-1 = a_1, \cdots, a_n+b_n-1 = a_n \implies b_1 = 1, \cdots , b_n = 1 \implies \vec{0} = (1, \dots, 1)               
            \end{math}
        \end{center}
        Així podem veure que es compleix la propietat del vector nul i aquest \'es ${\vec{0} = (1,\dots,1)}$
        
    \subsection*{4. Suma de l'invers}
        La quarta propietat que hem de demostrar \'es que ${\forall u \in \mathbb{R}^n}$, ${u +^* (-1)\cdot^*u = \vec{0}}$. Per demostrar-ho aplicarem la deficici\'o de ${+^*}$ i de ${\cdot^*}$:
        \begin{center}
            Sigui ${u = (a_1,\dots,a_n)}$ \\
            \begin{math}
                u+^*(-1)\cdot^*u = u +^* (-1\cdot(a_1-1)+1,\dots,-1\cdot(a_n-1)+1) = u +^* (-a_1+1+1,\dots,a_n+1+1) = u +^* (2-a_1,\dots,2-a_n) = (a_1+(2-a_1)-1,\dots,a_n+(2-a_n)-1) = (1,\dots,1) = \vec{0}
            \end{math}
        \end{center}

    \subsection*{5. Propietat distributiva de ${\cdot^*}$}
        

\end{document}  