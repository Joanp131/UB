\documentclass[a4paper, 11pt]{article}

\usepackage[utf8]{inputenc}
\usepackage{amsmath}
\usepackage{amssymb}
\usepackage[catalan]{babel}
\usepackage{fancyhdr}
\usepackage[
    left=0.90in,%
    right=0.90in,%
    top=1.0in,%
    bottom=1.0in,%
    paperheight=11in,%
    paperwidth=8.5in%
]{geometry}%

\pagestyle{fancy}
\fancyhf{}
\rhead{Joan Pau Condal Marco}
\lhead{Deures 13/03}
\rfoot{Pàg \thepage}

\begin{document}
    \begin{center}
        \Large
        \textbf{
            Homework 13/03\\
            Joan Pau Condal Marco\\
        }
    \end{center}

    \noindent\textbf{\large \underline{Exercici 1:} \normalsize Demostra que \emph{f*} es lineal.}\\\\
    Sigui $ f: E \longrightarrow F $ una aplicació lineal, definim $f^*$ de la seg\"uent manera:
    \begin{align*}
        f^*: F^* &\longrightarrow E^*\\
        \omega &\mapsto \omega \circ f
    \end{align*}
    Donada la definici\'o de $ f^* $, demostrarem les dues propietats per veure que l'aplicaci\'o \'es lineal.\\\\
    Siguin $ w,v \in F^* $ dues aplicacions lineals, sabem que:
    \begin{gather*}
        f^* (w+v) = (w+v) \circ f = (w+v)(f) =\\
        w(f) + v(f) = w \circ f + v \circ f = f^*(w) + f^*(v)
    \end{gather*}
    D'on queda demostrada la primera propietat de la linealitat.\\\\
    Per la segona propietat, considerarem $ \alpha \in \mathbb{R} \text{ i } w \in F^* $ aplicaci\'o lineal. Aleshores, per definici\'o de $ f^* $ sabem que:
    \begin{gather*}
        f^*(\alpha w) = (\alpha w) \circ f = (\alpha w)(f) =\\
        \alpha w(f) = \alpha (w \circ f) = \alpha f^*(w)
    \end{gather*}
    D'on queda demostrada la linealitat de $ f^* $.\\\\

    \noindent\textbf{\large \underline{Exercici 2:} \normalsize Demostra el seg\"uent corol·lari:
    $$
        C(\mathcal{B}^*_2, \mathcal{B}^*_1) = \left( C(\mathcal{B}_1, \mathcal{B}_2) \right)^t
    $$
    On C \'es la matriu de canvi de base; i $\mathbf{\mathcal{B}_1, \mathcal{B}_2}$ s\'on dues bases de l'espai \emph{E} de dimensi\'o finita.
    }\\\\
    Recordem que si tenim l'aplicaci\'o lineal $f:E \longrightarrow F$ i definim $ f^*: F^* \longrightarrow E^* $, aleshores
    \begin{gather}
        M_{\mathcal{B}_{F^*}\mathcal{B}_{E^*}}(f^*) = \left( M_{\mathcal{B}_E\mathcal{B}_F}(f) \right)^t
    \end{gather}
    Definim dues bases d'un espai \emph{E} de dimensi\'o finita i una base dual per cada base de \emph{E}
    $$
        \begin{array}{l r}
            \mathcal{B}_1 = \left\{ v_1,\dots,v_n \right\} & \mathcal{B}^*_1 = \left\{ v^*_1,\dots,v^*_n \right\}\\
            \mathcal{B}_2 = \left\{ u_1,\dots,u_n \right\} & \mathcal{B}^*_2 = \left\{ u^*_1,\dots,u^*_n \right\}
        \end{array}
    $$
    Definim ara l'endomorfisme \emph{f} de \emph{E} de la seg\"uent manera:
    \begin{gather*}
        f: E \longrightarrow E\\
        f(v_i) \mapsto u_i, \forall i = 1,\dots,n
    \end{gather*}
    D'aquesta manera, \emph{f} \'es l'aplicaci\'o de canvi de base de $\mathcal{B}_1$ a $\mathcal{B}_2$. Si aconseguim demostrar que $ f^* $ \'es la funci\'o de canvi de base de $ \mathcal{B}^*_2 $ a $ \mathcal{B}^*_1 $, per (1) haurem demostrat el corol·lari.\\\\
    Per definici\'o, $ f^* $ ser\`a l'endomorfisme de $ E^* $ tal que
    \begin{align*}
        f^*: E^* &\longrightarrow E^*\\
        \omega &\mapsto \omega \circ f
    \end{align*}
    Per demostrar que $ f^* $ \'es la funci\'o de canvi de base, hem de demostrar que
    $$
        f^*(u^*_i) = v^*_i \text{, } \forall i = 1,\dots,n
    $$
    Per la definici\'o de $ f^* $ sabem que
    \begin{gather}
            f^*(u^*_i)(v_j) = (u^*_i \circ f)(v_j) =
            u^*_i(f(v_j)) = u^*_i(u_j)\text{, } \forall i,j = 1, \dots, n
    \end{gather}
    I de la igualtat (2) podem veure que
    \begin{gather*}
        (u^*_i \circ f)(v_j) = 
        \begin{cases}
            1 \text{ si } i = j\\
            0 \text{ si } i \neq j
        \end{cases}
        \implies (u^*_i \circ f) = v^*_i, \forall i = 1,\dots,n
    \end{gather*}
    D'on conclu\"im que $ f^*(u^*_i)=v^*_i, \forall i = 1,\dots,n $; que significa que $ f^* $ \'es la funci\'o de canvi de base $\mathcal{B}_2^*$ a $ \mathcal{B}_1^* $, demostrant aix\'i el corol·lari.\\\\

    \noindent\textbf{ \large \underline{Exercici 3:}
    \normalsize
    Si $\mathcal{B}$ \'es una base de \emph{E}, $\mathcal{B}^*$ la seva base dual i $\mathcal{B}^{**}$ la base dual de la base dual, demostra que
    $$
        M_{\mathcal{B}\mathcal{B}^{**}}(\Psi) = \mathbf{I}
    $$
    la matriu identitat
    }\\\\
    Recordem de teoria les aplicacions $\psi_u$ i $\Psi$:
    \begin{align*}
        \psi_u: E &\longrightarrow \mathbb{R} & \Psi: E &\longrightarrow E^{**}\\
        \psi_u(\omega) &\mapsto \omega(u) & \Psi(u) &:= \psi_u\\
        \psi_u &\in E^{**}\\
    \end{align*}
    Definim les bases dels tres espais de la seg\"uent manera:
    \begin{align*}
        \mathcal{B} &= \left\{ u_1,\dots,u_n \right\}\\
        \mathcal{B}^* &= \left\{ v_1,\dots,v_n \right\}\\
        \mathcal{B}^{**} &= \left\{ w_1,\dots,w_n \right\}
    \end{align*}
    Per la definici\'o de matriu d'aplicaci\'o lineal, sabem que la columna \emph{i} de la matriu $ M_{\mathcal{B}\mathcal{B}^{**}}(\Psi) $ ser\`a $ \Psi(u_i) = \psi_{u_i} $. Aplicant la definici\'o de $ \psi_{u_i} $ sabem que $ \psi_{u_i}(v_j) = v_j(u_i), \forall j=1,\dots,n $. Per tant
    \begin{gather*}
        \psi_{u_i}(v_j) = v_j(u_i) = 
        \begin{cases}
            1 \text{ si } i = j\\
            0 \text{ si } i \neq j
        \end{cases}
        \implies \psi_{u_i} = w_i
    \end{gather*}
    Finalment, si representem cada $ \psi_{u_i} \text{, } i = 1,\dots,n $ en coordenades de $ \mathcal{B}^{**} $ obtenim:
    \begin{align*}
        (\psi_{u_1})_{\mathcal{B}^{**}} &= (1, 0, \dots, 0)\\
        (\psi_{u_2})_{\mathcal{B}^{**}} &= (0, 1, 0, \dots, 0)\\
        &\vdots\\
        (\psi_{u_n})_{\mathcal{B}^{**}} &= (0, \dots, 0, 1)
    \end{align*}
    I al construir la matriu de $ \Psi $ ens queda
    \begin{gather*}
        M_{\mathcal{B}\mathcal{B}^{**}}(\Psi) = \left[ \psi_{u_1}, \dots, \psi_{u_n} \right] =
        \begin{bmatrix}
            1 & 0 & 0 & \cdots & 0 \\
            0 & 1 & 0 & \cdots & 0 \\
            0 & 0 & 1 & \cdots & 0 \\
            \vdots & \vdots & \vdots & \ddots & \vdots \\
            0 & 0 & 0 & \cdots & 1
        \end{bmatrix}
        = \mathbf{I} 
    \end{gather*}

\end{document}